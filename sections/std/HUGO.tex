\section{Capitulo Hugo}
Electromagnetismo: Es la rama de la física que estudia las interacciones entre las partículas con carga eléctrica y los campos eléctricos y magnéticos. Es una de las cuatro fuerzas fundamentales de la naturaleza.


\[
\vec{F} = k_e \frac{q_1 q_2}{r^2} \hat{r}
\]
Donde \( k_e = \frac{1}{4\pi\varepsilon_0} \), y \( \hat{r} \) es el vector unitario entre las cargas.

\subsection*{2. Campo Eléctrico}

\[
\vec{E} = \frac{\vec{F}}{q} = k_e \frac{q}{r^2} \hat{r}
\]

\subsection*{3. Ley de Gauss (Campo eléctrico)}

\[
\oint_{\partial V} \vec{E} \cdot d\vec{A} = \frac{Q_{\text{int}}}{\varepsilon_0}
\]

\subsection*{4. Potencial Eléctrico}

\[
V = - \int \vec{E} \cdot d\vec{l}
\]

\[
\vec{E} = - \nabla V
\]

\subsection*{5. Energía Potencial Eléctrica}

\[
U = qV
\]

\subsection*{6. Ley de Biot-Savart}

\[
d\vec{B} = \frac{\mu_0}{4\pi} \frac{I\, d\vec{l} \times \hat{r}}{r^2}
\]

\subsection*{7. Ley de Ampère}

\[
\oint \vec{B} \cdot d\vec{l} = \mu_0 I_{\text{int}}
\]

\subsection*{8. Ley de Gauss para el Magnetismo}

\[
\oint_{\partial V} \vec{B} \cdot d\vec{A} = 0
\]

\subsection*{9. Ley de Faraday (Inducción electromagnética)}

\[
\mathcal{E} = -\frac{d\Phi_B}{dt}
\]

\[
\Phi_B = \int \vec{B} \cdot d\vec{A}
\]

\subsection*{10. Ley de Lenz (Sentido de la corriente inducida)}

La corriente inducida se opone a la variación del flujo magnético:
\[
\mathcal{E} = -\frac{d\Phi_B}{dt}
\]

\subsection*{11. Ley de Lorentz (Fuerza sobre una carga en campos \(\vec{E}\) y \(\vec{B}\))}

\[
\vec{F} = q (\vec{E} + \vec{v} \times \vec{B})
\]

\subsection*{12. Ecuaciones de Maxwell (Forma diferencial)}

\begin{align*}
\nabla \cdot \vec{E} &= \frac{\rho}{\varepsilon_0} \quad &\text{(Ley de Gauss)} \\
\nabla \cdot \vec{B} &= 0 \quad &\text{(No hay monopolos magnéticos)} \\
\nabla \times \vec{E} &= -\frac{\partial \vec{B}}{\partial t} \quad &\text{(Ley de Faraday)} \\
\nabla \times \vec{B} &= \mu_0 \vec{J} + \mu_0 \varepsilon_0 \frac{\partial \vec{E}}{\partial t} \quad &\text{(Ley de Ampère-Maxwell)}
\end{align*}

\subsection*{13. Energía almacenada en el campo electromagnético}

\[
u = \frac{1}{2} \left( \varepsilon_0 E^2 + \frac{1}{\mu_0} B^2 \right)
\]

\subsection*{14. Vector de Poynting (flujo de energía electromagnética)}

\[
\vec{S} = \frac{1}{\mu_0} (\vec{E} \times \vec{B})
\]
