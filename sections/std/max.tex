\section{capitulo max}
La Geología de Minas es la rama de la geología que se enfoca en la aplicación de principios y técnicas geológicas para el descubrimiento, exploración, evaluación, desarrollo y explotación de yacimientos de recursos minerales de manera eficiente y segura. Se apoya en disciplinas como la geotecnia y geoquímica, e integra conocimientos de geología estructural, petrología y mineralogía para entender la composición, estructura, continuidad y relaciones espaciales de las rocas que albergan los recursos, lo que es crucial para el diseño de minas y la optimización de la producción.

ECUACION DE BERNULL
\[
i\hbar \frac{\partial \Psi(\mathbf{r},t)}{\partial t}
=
\left(
-\frac{\hbar^{2}}{2m}\nabla^{2} + V(\mathbf{r},t)
\right)\Psi(\mathbf{r},t)
\]
I\[
i\hbar \frac{\partial \Psi(\mathbf{r},t)}{\partial t}
=
\left(
-\frac{\hbar^{2}}{2m}\nabla^{2} + V(\mathbf{r},t)
\right)\Psi(\mathbf{r},t)
\]
\[
i\hbar \frac{\partial \Psi(\mathbf{r},t)}{\partial t}
=
\left(
-\frac{\hbar^{2}}{2m}\nabla^{2} + V(\mathbf{r},t)
\right)\Psi(\mathbf{r},t)
\]
 
\[
p + \tfrac{1}{2}\rho v^{2} + \rho g h = \text{constante}
\]
\[
p_1 + \tfrac{1}{2}\rho v_1^{2} + \rho g h_1
=
p_2 + \tfrac{1}{2}\rho v_2^{2} + \rho g h_2 + \rho g h_f
\]
\[
z(x) \, e^{\int (1-n)P(x)\,dx}
=
\int (1-n)Q(x)\, e^{\int (1-n)P(x)\,dx}\,dx + C
\]
\[
\text{y luego } y(x) = \bigl(z(x)\bigr)^{\frac{1}{1-n}}
\]
\[
z(x) \, e^{\int (1-n)P(x)\,dx}
=
\int (1-n)Q(x)\, e^{\int (1-n)P(x)\,dx}\,dx + C
\]
\[
\text{y luego } y(x) = \bigl(z(x)\bigr)^{\frac{1}{1-n}}
\]
\[
i\hbar \frac{\partial \Psi(\mathbf{r},t)}{\partial t}
=
\left(
-\frac{\hbar^{2}}{2m}\nabla^{2} + V(\mathbf{r},t)
\right)\Psi(\mathbf{r},t)
\]
\[
i\hbar \frac{\partial \Psi(\mathbf{r},t)}{\partial t}
=
\left(
-\frac{\hbar^{2}}{2m}\nabla^{2} + V(\mathbf{r},t)
\right)\Psi(\mathbf{r},t)
\]
