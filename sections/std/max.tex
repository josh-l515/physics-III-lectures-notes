\section{capitulo max}
La Geología de Minas es la rama de la geología que se enfoca en la aplicación de principios y técnicas geológicas para el descubrimiento, exploración, evaluación, desarrollo y explotación de yacimientos de recursos minerales de manera eficiente y segura. Se apoya en disciplinas como la geotecnia y geoquímica, e integra conocimientos de geología estructural, petrología y mineralogía para entender la composición, estructura, continuidad y relaciones espaciales de las rocas que albergan los recursos, lo que es crucial para el diseño de minas y la optimización de la producción. 