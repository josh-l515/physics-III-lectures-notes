\section{coordenadas cartesianas}
El sistema de coordenadas cartesianas se presentó en la sección 3-1, donde lo utilizamos para ilustrar muchas de las leyes del álgebra vectorial. En vez de repetirlas para el sistema cartesiano, las hemos resumido para facilitar su acceso en la tabla 3-1. 

En cálculo diferencial, con frecuencia se trabaja con cantidades diferenciales. La longitud diferencial en coordenadas cartesianas es un vector (véase la figura 3-8) definido como:

\begin{equation}
    d\mathbf{l} = \hat{\mathbf{x}}\,dl_x + \hat{\mathbf{y}}\,dl_y + \hat{\mathbf{z}}\,dl_z = \hat{\mathbf{x}}\,dx + \hat{\mathbf{y}}\,dy + \hat{\mathbf{z}}\,dz
    \label{eq:3.34}
\end{equation}

donde $dl_x = dx$ es una longitud diferencial a lo largo de $\hat{\mathbf{x}}$ y definiciones similares se aplican a $dl_y = dy$ y $dl_z = dz$.

