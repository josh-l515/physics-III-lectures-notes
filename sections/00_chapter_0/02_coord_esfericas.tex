\section{ Coordenadas Esfericas }
\begin{itemize}
    \item   en el sistema de coordenadas esfericas la ubicacion de un punto en el espacio se especifica unicamente por las variables R, R, u y f, como se
 indica en la figura  
\end{itemize}
\subsection*{3.2.3 Coordenadas esféricas}

En el sistema de coordenadas esféricas, la ubicación de un punto en el espacio se especifica únicamente por las variables \( R \), \( \theta \) y \( \phi \), como se indica en la figura 3-13. La coordenada \( R \), que en ocasiones se llama coordenada de rango, describe una esfera de radio \( R \) con centro en el origen. El ángulo cenit \( \theta \) se mide a partir del eje \( z \) positivo y describe una superficie cónica con su vértice en el origen y el ángulo azimutal \( \phi \) es el mismo como en el sistema de coordenadas cilíndricas.

Los rangos de \( R \), \( \theta \) y \( \phi \) son \( 0 \leq R < \infty \), \( 0 \leq \theta \leq \pi \) y \( 0 \leq \phi < 2\pi \). Los vectores base \( \hat{\mathbf{R}} \), \( \hat{\boldsymbol{\theta}} \) y \( \hat{\boldsymbol{\phi}} \) obedecen las relaciones cíclicas de la mano derecha:

\begin{equation}
\hat{\mathbf{R}} \times \hat{\boldsymbol{\theta}} = \hat{\boldsymbol{\phi}}, \quad
\hat{\boldsymbol{\theta}} \times \hat{\boldsymbol{\phi}} = \hat{\mathbf{R}}, \quad
\hat{\boldsymbol{\phi}} \times \hat{\mathbf{R}} = \hat{\boldsymbol{\theta}}. \tag{3.45}
\end{equation}

