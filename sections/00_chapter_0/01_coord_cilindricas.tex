\section{Cordenadas Cilindricas}
Un sistema de coordenadas cilíndricas es útil para resolver problemas que tienen simetría cilíndrica, como calcular la capacitancia por unidad de longitud de una línea de transmisión coaxial. La localización de un punto en el sistema de coordenadas cilíndricas se define por tres variables: \( r \), \( \phi \) y \( z \), como se ilustra en la figura correspondiente.

\begin{itemize}
    \item La coordenada \( r \) es la distancia radial en el plano \( x\text{-}y \).
    \item \( \phi \) es el ángulo azimutal medido desde el eje \( x \) positivo hacia el eje \( y \), como se definió previamente en el sistema de coordenadas cartesianas.
    \item Los puntos con \( 0 \leq r < \infty \), \( 0 \leq \phi < 2\pi \) y \( -\infty < z < \infty \) se localizan en el espacio.
\end{itemize}

El punto \( P \) se localiza en la intersección de tres superficies:
\begin{itemize}
    \item Un cilindro de radio \( r \).
    \item Un plano que contiene el eje \( z \) y que forma un ángulo \( \phi \) con el eje \( x \).
    \item Un plano horizontal \( z = z_0 \).
\end{itemize}

El vector de posición \( \vec{r} \) se extiende hacia el punto desde el origen. Los vectores base mutuamente perpendiculares \( \vec{e}_r \), \( \vec{e}_\phi \) y \( \vec{e}_z \) están orientados a lo largo de \( r \), \( \phi \) y \( z \), respectivamente:
\begin{itemize}
    \item \( \vec{e}_r \) apunta radialmente hacia afuera.
    \item \( \vec{e}_\phi \) es perpendicular a \( \vec{e}_r \) en el plano \( x\text{-}y \) y apunta en la dirección de aumento de \( \phi \).
    \item \( \vec{e}_z \) apunta a lo largo del eje \( z \).
\end{itemize}
Los vectores base \( \vec{e}_r \), \( \vec{e}_\phi \) y \( \vec{e}_z \) son funciones de \( \phi \), mientras que el vector de posición \( \vec{r} \) y las coordenadas \( r \), \( \phi \) y \( z \) de un punto \( P \) también son funciones de \( \phi \).
